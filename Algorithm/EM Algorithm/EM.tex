\documentclass[UTF8,12pt]{ctexart}

\usepackage{geometry}%排版

%%%%%%%%%%%%数学%%%%%%%%%%%%%%%%
\usepackage{amsmath}
\usepackage{amsfonts}
\usepackage{amssymb}%公式
\usepackage{amsthm}%定理环境
%\usepackage{ntheorem}%定理环境,使用这个会使\maketitle版式出问题
\usepackage{mathrsfs}
\usepackage{witharrows}

\numberwithin{equation}{section}%对公式以节{section}为基础进行编号.变成(1.1.1)有chapter才有1.1.1,不然只有section是1.1
%\theoremstyle{plain}%定理用latex默认的版式
\newtheorem{thm}{Theorem}[section]
%\theoremstyle{definition}%定义用definition格式
\newtheorem{defn}{Definition}
%\theoremstyle{remark}%用remark格式
\newtheorem{lemma}[thm]{lemma}
\newtheorem{example}{Example}[section]
\newtheorem{prop}{命题}[section]   % 定义新的命题环境
\usepackage{bm}%加粗

%%%%%%%%%%%%表格%%%%%%%%%%%%
\usepackage{multirow}%表格列合并宏包,\multirow命令.
\usepackage{tabularx}%表格等宽,\begin{tabularx}命令.
	
%%%%%%%%%%%%图片%%%%%%%%%%%%	
\usepackage{graphicx}
\usepackage{float}
\usepackage{subfigure} %插入多图时用子图显示的宏包
	
%%%%%%%%%%%%算法%%%%%%%%%%%%
%\usepackage{algorithm,algorithmic}%算法
\usepackage{algorithm}
%\usepackage{algorithmic}
\usepackage{algpseudocode}
	
%%%%%%%%%%%%代码%%%%%%%%%%%%	
%\usepackage{minted}
	
%%%%%%%%%%%%盒子%%%%%%%%%%%%%
%main
\usepackage{varwidth}
\usepackage{tikz}
\usetikzlibrary{calc} % 引入 calc 库以使用高级坐标计算
%\usepackage{xeCJK}
\usepackage[most]{tcolorbox}
\tcbuselibrary{xparse,hooks,skins,breakable}
\tcbuselibrary{listings}
\usepackage{fontawesome}
\usepackage{mdframed}
\usepackage{xcolor}	
\input{box.tex}	

\definecolor{citecolor}{RGB}{60,120,216}
\definecolor{urlcolor}{RGB}{60,120,216}
\definecolor{linkcolor}{RGB}{60,120,216}

\usepackage{hyperref}%可以生成pdf书签,可以跳转
\hypersetup{
	colorlinks=true,%引用部分会显示颜色
	linkcolor=linkcolor, %
	urlcolor=urlcolor,%
	allcolors=blue,
	citecolor=citecolor,
	pdftitle={EM Algorithm},%这个命令会使用pdf阅读器打开时左上角显示这个内容
}

	
%\tcbuselibrary{skins, breakable}% 支持文本框跨页
	
%%%%%%%%%%%%参考文献%%%%%%%%%%%%%
\usepackage[round]{natbib}
	
%%%%%%%%%%%%其他%%%%%%%%%%%%%
\usepackage[english]{babel}% 载入美式英语断字模板
\usepackage{caption}
	
	
\usepackage{marginnote}
%\usepackage{xcolor} % 用于颜色设置
\usepackage{lipsum} % 生成示例文本
\usepackage{ifthen} % 提供条件判断命令
\usepackage{graphicx}
	
% 定义边注颜色
\definecolor{mynotecolor}{rgb}{0.6, 0.4, 0.2} % 棕色
	
\newcommand{\mymarginnote}[1]{
	\ifthenelse{\isodd{\value{page}}}
	{\normalmarginpar\marginnote{\textcolor{mynotecolor}{#1}}} % 奇数页的边注
	{\reversemarginpar\marginnote{\textcolor{mynotecolor}{#1}}} % 偶数页的边注
}
	
	
%%%%%%%%%%%%标题页%%%%%%%%%%%%%
\title{EM Algorithm}
\author{Renhe W.}
\date{ }
	
\begin{document}
	\captionsetup[figure]{labelfont={bf},labelformat={default},labelsep=period,name={图}}%设置图注
	%\newgeometry{left=3cm,right=3cm,top=3cm,bottom=3cm}
	%\pagestyle{empty} % 可选,如果您不想在标题页和目录页显示页眉和页脚
		
	\maketitle
	\tableofcontents%目录
	\listoffigures%图片目录
	\listoftables%表格目录
	\newpage
	\kaishu
	
	% 恢复双面布局
	%\restoregeometry
	%\pagestyle{headings} % 恢复页眉和页脚
	
		
	%------------------------------------------- 
	\section{EM算法}
	EM算法的普遍应用主要归功于DLR \citep{dempster1977maximum},他们在研究中还提供了许多其适用性示例,并在相当普遍的条件下确立了其收敛性和其他基本性质。对于一个具有隐藏状态的系统,我们假设观测序列为$\mathbf{y}=\{y_1,y_2,\dots,y_T\}$,隐藏状态为$\mathbf{s}=\{s_1,s_2,\dots,s_T\}$,其中$T$是观测序列时长. 定义非完全信息似然函数(incomplete-data likelihood function)为$L(\mathbf{y}\mid \mathbf{\theta})$,其中$\boldsymbol{\theta}$是模型参数,则有:
	\begin{DispWithArrows}[tikz=blue,tagged-lines=last]
		\log L(\mathbf{y} \mid \boldsymbol{\theta}) & =\log f(\mathbf{y} \mid \boldsymbol{\theta}) \Arrow{联系完全似然}\\
		& =\log f(\mathbf{y} \mid \boldsymbol{\theta}) \cdot \frac{f(\mathbf{y}, \mathbf{s} \mid \boldsymbol{\theta})}{f(\mathbf{y}, \mathbf{s} \mid \boldsymbol{\theta})} \\
		& =\log  f(\mathbf{y}, \mathbf{s} \mid \boldsymbol{\theta})\cdot \frac{f(\mathbf{y} \mid \boldsymbol{\theta})}{f(\mathbf{y}, \mathbf{s} \mid \boldsymbol{\theta})} \Arrow{条件概率}\\
		& =\log\underbrace{f(\mathbf{y}, \mathbf{s} \mid \boldsymbol{\theta})}_{\text{完全信息似然函数}}-\log f(\mathbf{s} \mid \mathbf{y}, \theta)\label{incomplete data decomposition}
	\end{DispWithArrows}
	同时对\eqref{incomplete data decomposition}式子两边关于$\mathbf{y}$和$\boldsymbol{\theta}'$取期望可以得到:
	%\vspace{-20pt}
    \begin{DispWithArrows}[tikz=blue,tagged-lines=last]
    	\sum_s \log L(\mathbf{y} \mid \boldsymbol{\theta}) \cdot f\left(\mathbf{s}\mid \mathbf{y}, \boldsymbol{\theta}^{\prime}\right) & =\sum_s \log f(\mathbf{y}, \mathbf{s} \mid \boldsymbol{\theta}) \cdot f\left(\mathbf{s} \mid \mathbf{y}, \boldsymbol{\theta}^{\prime}\right)-\sum_s \log f(\mathbf{s} \mid \mathbf{y}, \boldsymbol{\theta}) \cdot f\left(\mathbf{s} \mid \mathbf{y}, \boldsymbol{\theta}^{\prime}\right) \\
    	\Rightarrow \quad  \log L(\mathbf{y} \mid \boldsymbol{\theta}) \cdot \underbrace{\sum_sf\left(\mathbf{s} \mid \mathbf{y}, \boldsymbol{\theta}^{\prime}\right) }_{=1}  & =E\left(\log f(\mathbf{y}, \mathbf{s} \mid \boldsymbol{\theta}) \mid \mathbf{y}, \boldsymbol{\theta}^{\prime}\right)-E\left(\log f(\mathbf{s} \mid \mathbf{y}, \boldsymbol{\theta}) \mid \mathbf{y}, \boldsymbol{\theta}^{\prime}\right) \\
    	\Rightarrow \quad		\log L(\mathbf{y} \mid \boldsymbol{\theta})		& =Q\left(\boldsymbol{\theta}, \boldsymbol{\theta}^{\prime}\right)-H\left(\boldsymbol{\theta}, \boldsymbol{\theta}^{\prime}\right).\label{likelihood relation with Q function}
    \end{DispWithArrows}
	
	\subsection{EM Algorithm}
	EM算法(期望最大化算法)是一种用于含有隐变量的统计数据估计的迭代算法。它通过交替执行两个步骤:期望步骤(E-step)和最大化步骤(M-step)——来找到参数的最大似然估计或最大后验估计.EM算法的目的是最大化非完全信息似然函数$L(\mathbf{y} \mid \boldsymbol{\theta})$。在这种情况下,EM算法的两个步骤可以这样表述:
	
	\begin{ascolorbox13}[步骤]{EM算法}
		\ascboxB*[C]{\textbf{1. E-Step(期望步骤):}}
		在第$k$次迭代中,E-step的目的是计算在当前参数估计$\boldsymbol{\theta}^{(k)}$的条件下,完全数据对数似然函数$\log L(\mathbf{y}, s \mid \boldsymbol{\theta})$的期望值。这一步骤可以表示为:
		\[ Q\left(\boldsymbol{\theta} ; \boldsymbol{\theta}^{(k)}\right) = E\left(\log f(\mathbf{y}, \mathbf{s} \mid \boldsymbol{\theta}) \mid \mathbf{y}, \boldsymbol{\theta}^{(k)}\right) \]
		其中$Q(\boldsymbol{\theta}; \boldsymbol{\theta}^{(k)})$是在给定观测数据$\mathbf{y}$和当前参数估计$\boldsymbol{\theta}^{(k)}$的情况下,关于隐状态的完全数据对数似然的期望.
		
		\ascboxB*[C]{\textbf{2. M-Step(最大化步骤):}}
		在M-step中,目标是找到参数$\boldsymbol{\theta}$的新估计值$\boldsymbol{\theta}^{(k+1)}$,使得$Q(\boldsymbol{\theta}; \boldsymbol{\theta}^{(k)})$最大化. 数学上表示为:
		\[ \boldsymbol{\theta}^{(k+1)} = \arg \max_{\boldsymbol{\theta}} Q(\boldsymbol{\theta}; \boldsymbol{\theta}^{(k)}) \]
		选择$\boldsymbol{\theta}^{(k+1)}$作为参数$\boldsymbol{\theta}$的新估计值,使得$Q$函数在这一点上达到最大.
		
		通过交替执行这两个步骤,EM算法在每次迭代中更新参数$\boldsymbol{\theta}$的估计值,直到似然函数$L(\boldsymbol{\theta})$的值收敛到一个固定值,或达到预定的迭代次数。这个过程保证了每次迭代后,不完全数据的对数似然函数$L(\boldsymbol{\theta})$不会减少,从而实现对参数的有效估计.
	\end{ascolorbox13}
	
	

	

	
	\subsection{How it works?}
	DLR(Dempster, Laird, and Rubin)在他们的工作中证明了,在一定条件下,EM算法可以保证每次迭代后,不完全数据的对数似然函数$L(\mathbf{y}\mid \boldsymbol{\theta})$不会减少. 这意味着,通过EM算法得到的参数估计序列将收敛到一个局部最大值. 
	
	\begin{prop}[单调不减(MONOTONICITY)]\label{MONOTONICITY}
		$L(\mathbf{y}\mid \boldsymbol{\theta}^{(k+1)})\ge L(\mathbf{y}\mid \boldsymbol{\theta}^{(k)})$.
	\end{prop}
	
	\begin{proof} 根据\eqref{likelihood relation with Q function}式,有
		$$
		\begin{aligned}
			& \log L\left(\mathbf{y} \mid \boldsymbol{\theta}^{(k+1)}\right)-\log L\left(y \mid \boldsymbol{\theta}^{(k)}\right) \\
			= & \underbrace{\left\{Q\left(\boldsymbol{\theta}^{(k+1)}, \boldsymbol{\theta}^{(k)}\right)-Q\left(\boldsymbol{\theta}^{(k)}, \boldsymbol{\theta}^{(k)}\right)\right\}}_{\operatorname{term1}}
			\underbrace{-
			\left\{H\left(\boldsymbol{\theta}^{(k+1)} , \boldsymbol{\theta}^{(k)}\right)-H\left(\boldsymbol{\theta}^{(k)}, \boldsymbol{\theta}^{(k)}\right)\right\} }_{\operatorname{term2}} \\
			= & \operatorname{term1} +\operatorname{term2},
		\end{aligned}
		$$
		其中$\operatorname{term1}$中的Q函数每一步进行的过程中都有求导梯度更新,所以有$Q\left(\boldsymbol{\theta}^{(k+1)}, \boldsymbol{\theta}^{(k)}\right)\ge Q\left(\boldsymbol{\theta}^{(k)}, \boldsymbol{\theta}^{(k)}\right)$,现在的主要工作为$\operatorname{term2}$,对于$\boldsymbol{\theta}\in\Omega$,若$\operatorname{term2}$每次更新都是负值,i.e. $H\left(\boldsymbol{\theta}^{(k+1)} , \boldsymbol{\theta}^{(k)}\right)\le H\left(\boldsymbol{\theta}^{(k)}, \boldsymbol{\theta}^{(k)}\right)$ ,则命题\ref{MONOTONICITY}成立,下面给出其中一个证明:
		
		对于任意参数$\boldsymbol{\theta}$,有:
		$$
		\begin{WithArrows}[tikz=blue]
			H(\boldsymbol{\theta} , \boldsymbol{\theta}^{(k)})&- H(\boldsymbol{\theta}^{(k)}, \boldsymbol{\theta}^{(k)})\\
			&= E\left(\log f(\mathbf{s} \mid \mathbf{y}, \boldsymbol{\theta}) \mid \mathbf{y}, \boldsymbol{\theta}^{(k)}\right)-E\left(\log f(\mathbf{s} \mid \mathbf{y}, \boldsymbol{\theta}^{(k)}) \mid \mathbf{y}, \boldsymbol{\theta}^{(k)}\right)\\
			&= E\left(\log f(\mathbf{s} \mid \mathbf{y}, \boldsymbol{\theta})/f(\mathbf{s} \mid \mathbf{y}, \boldsymbol{\theta}^{(k)}) \mid \mathbf{y}, \boldsymbol{\theta}^{(k)}\right)\Arrow{Jensen inequality}\\
			&\le \log[E\left( f(\mathbf{s} \mid \mathbf{y}, \boldsymbol{\theta})/f(\mathbf{s} \mid \mathbf{y}, \boldsymbol{\theta}^{(k)}) \mid \mathbf{y}, \boldsymbol{\theta}^{(k)}\right)]\\
			&=log\sum_s \frac{f(\mathbf{s} \mid \mathbf{y}, \boldsymbol{\theta})}{f(\mathbf{s} \mid \mathbf{y}, \boldsymbol{\theta}^{(k)})}\cdot f(\mathbf{s} \mid \mathbf{y}, \boldsymbol{\theta}^{(k)})\\
			&= log\sum_s f(\mathbf{s} \mid \mathbf{y}, \boldsymbol{\theta})= \log 1 = 0.
		\end{WithArrows}
		$$
		综上,对数似然函数$L(\mathbf{y}\mid \boldsymbol{\theta})$通过EM算法更新迭代一直单调不减.	
	\end{proof}
	\begin{marker}
		对于$H\left(\boldsymbol{\theta}^{(k+1)} , \boldsymbol{\theta}^{(k)}\right)\le H\left(\boldsymbol{\theta}^{(k)}, \boldsymbol{\theta}^{(k)}\right)$,还可以运用KL散度进行证明,KL散度是衡量两个概率分布间差异的度量,定义为
		$$D_{\mathrm{KL}}(P \| Q)=\sum_x P(x) \log \frac{P(x)}{Q(x)}$$
		其中, $ P $ 和 $ Q $ 是两个概率分布.
	\end{marker}
	通过命题\ref{MONOTONICITY}以及相应的证明,对数似然函数$L(\mathbf{y}\mid \boldsymbol{\theta})$通过EM算法更新迭代一直单调不减,下面给出算法收敛的证明.
	\begin{ascolorbox3}{\citet{wu1983convergence}}
		\citet{wu1983convergence}为确保似然序列$\{L(\mathbf{y}\mid \boldsymbol{\theta}^{(k)})\}$收敛到一个稳定值,给出以下几个条件:
		\begin{itemize}
			\item $\Omega$ 是$d$维欧几里得空间$\mathbb{R}^d$的中的子集.
			\item  对于 $\forall L\left(\mathbf{y} \mid \boldsymbol{\theta_0}\right)>-\infty ,\, \Omega_{\boldsymbol{\theta_0}}=\left\{\boldsymbol{\theta} \in \Omega: L(\mathbf{y} \mid \boldsymbol{\theta}) \ge L\left(\mathbf{y} \mid \boldsymbol{\theta_0}\right)\right\} $ 是一个紧集. 
			\item $L(\mathbf{y}\mid \boldsymbol{\theta})$ 在 $\Omega$ 中连续,在 $\Omega$ 上可微.
		\end{itemize}
	\end{ascolorbox3}
	\begin{prop}[收敛性(CONVERGENCE)]\label{CONVERGENCE}
	似然序列$\{L(\mathbf{y}\mid \boldsymbol{\theta}^{(k)})\}$单调收敛到$L^{\star}=L(\mathbf{y}\mid \boldsymbol{\theta}^{\star})$.
	\end{prop}
	
	\begin{proof}
			似然序列$\{L(\mathbf{y}\mid \boldsymbol{\theta}^{(k)})\}$单调收敛到$L^{\star}=L(\mathbf{y}\mid \boldsymbol{\theta}^{\star})$,即
		
		$$
		\frac{\partial L(\mathbf{y}\mid \boldsymbol{\theta}^{\star})}{\partial \boldsymbol{\theta}^{\star}}=0 .
		$$	
		即也表示为
		$$
		\frac{\partial \log L(\mathbf{y}\mid \boldsymbol{\theta}^{\star})}{\partial \boldsymbol{\theta}^{\star}}=0 .
		$$
		这里假设 $L(y \mid \theta)$ 是单峰函数(在$\Omega$中,并可微),对\eqref{likelihood relation with Q function}式两边求导,有
		\begin{equation}\label{convergence Q and H}
			\frac{\partial \log L(\mathbf{y}\mid \boldsymbol{\theta})}{\partial\boldsymbol{\theta}}=\frac{\partial Q\left(\boldsymbol{\theta}, \boldsymbol{\theta}^{(k)}\right)}{\partial \boldsymbol{\theta}}-\frac{\partial H\left(\boldsymbol{\theta}, \boldsymbol{\theta}^{(k)}\right)}{\partial \boldsymbol{\theta}},
		\end{equation}
	由命题\ref{MONOTONICITY}的证明,有$H(\boldsymbol{\theta} , \boldsymbol{\theta}^{(k)})\le H(\boldsymbol{\theta}^{(k)}, \boldsymbol{\theta}^{(k)})$
	,此时可以理解为到达了平稳点. 则对于所有$\boldsymbol{\theta}\in\Omega$, 有:
	\begin{equation}\label{H function partial}
		\left.\frac{\partial H\left(\boldsymbol{\theta}, \boldsymbol{\theta}^{(k)}\right)}{\partial \boldsymbol{\theta}}\right|_{\boldsymbol{\theta}=\boldsymbol{\theta}^{(k)} }=0 .
	\end{equation}
	
	令$\boldsymbol{\theta_0}$是$\boldsymbol{\theta}$的任一值,将$\boldsymbol{\theta}^{(k)}=\boldsymbol{\theta_0}$放入\eqref{convergence Q and H}中,又根据\eqref{H function partial},可以得到
	\begin{equation}\label{partial Likelihood}
		\left.\frac{\partial \log L(\mathbf{y} \mid \boldsymbol{\theta})}{\partial \boldsymbol{\theta}}\right|_{\boldsymbol{\theta}=\boldsymbol{\theta_0}}=
		\left.\frac{\partial Q\left(\boldsymbol{\theta}, \boldsymbol{\theta}^{(k)}\right)}{\partial \boldsymbol{\theta}}\right|_{\boldsymbol{\theta}=\boldsymbol{\theta_0}}
	\end{equation}
	
	假设$\boldsymbol{\theta}=\boldsymbol{\theta}^{\star}$时,$\boldsymbol{\theta}^{\star}$是$\log L(\mathbf{y} \mid \boldsymbol{\theta})$的一个平稳点,由\eqref{partial Likelihood}得
	\begin{equation}
		\left.\frac{\partial \log L(\mathbf{y} \mid \boldsymbol{\theta})}{\partial \boldsymbol{\theta}}\right|_{\boldsymbol{\theta}=\boldsymbol{\theta}^{\star}}=
		\left.\frac{\partial Q\left(\boldsymbol{\theta}, \boldsymbol{\theta}^{(k)}\right)}{\partial \boldsymbol{\theta}}\right|_{\boldsymbol{\theta}=\boldsymbol{\theta}^{\star}}
	\end{equation}
	则若$Q\left(\boldsymbol{\theta}, \boldsymbol{\theta}^{\star}\right)$在$\boldsymbol{\theta}^{\star}\in\Omega$全局最优,则EM算法可以收敛到鞍点$\boldsymbol{\theta}^{\star}$.
	\end{proof}
	
	\subsection{Score Statistic}\label{Score Statistic}
	\begin{ascolorbox16}{Score Statistic}
		对数似然函数的梯度向量:
		$$S(\boldsymbol{\theta})=\frac{\partial \log f(\mathbf{y}|\boldsymbol{\theta})}{\partial \boldsymbol{\theta}} $$
		
		完全对数似然函数的梯度向量:
		$$S_c(\boldsymbol{\theta})=\frac{\partial \log f(\mathbf{y}, \mathbf{s}|\boldsymbol{\theta})}{\partial \boldsymbol{\theta}} $$
	\end{ascolorbox16}
	其中 $S(\boldsymbol{\theta})$ 可以通过 $S(\boldsymbol{\theta})$ 表示:
	
	\begin{DispWithArrows}[tikz=blue,tagged-lines=last]
	S(\boldsymbol{\theta})& =\frac{\partial \log f(\mathbf{y} \mid \boldsymbol{\theta})}{\partial \boldsymbol{\theta}}=\frac{\partial f(\mathbf{y} \mid \boldsymbol{\theta}) / \partial \boldsymbol{\theta}}{f(\mathbf{y} \mid \boldsymbol{\theta})} \\
	& =\sum \frac{\partial f(\mathbf{y}, \mathbf{s} \mid \boldsymbol{\theta})}{\partial \boldsymbol{\theta}} / f(\mathbf{y} \mid \boldsymbol{\theta}) \Arrow{加一个log变换} \\
	& = \sum {\color{blue}\frac{\partial \log f(\mathbf{y}, \mathbf{s} \mid \boldsymbol{\theta})}{\partial \boldsymbol{\theta}} \cdot \frac{f(\mathbf{y}, \mathbf{s} \mid \boldsymbol{\theta})}{f(\mathbf{y} \mid \boldsymbol{\theta})} }\\
	& =\sum \frac{\partial \log f(\mathbf{y}, \mathbf{s} \mid \boldsymbol{\theta})}{\partial \boldsymbol{\theta}} \cdot f(\mathbf{s} \mid \mathbf{y}, \boldsymbol{\theta}) \\
	& =E\{\frac{\partial \log f(\mathbf{y}, \mathbf{s} \mid \boldsymbol{\theta})}{\partial \boldsymbol{\theta}} \mid \mathbf{y}, \boldsymbol{\theta}\}\\
	&=E\{S_c(\boldsymbol{\theta}) \mid \mathbf{y}, \boldsymbol{\theta}\}.\label{S and conditional Sc}
\end{DispWithArrows}

i.e. $S(\boldsymbol{\theta})=\left.\frac{\partial Q\left(\boldsymbol{\theta}_0, \boldsymbol{\theta}\right)}{\partial \boldsymbol{\theta}_0}\right|_{\boldsymbol{\theta}_0=\boldsymbol{\theta} }$.
	
	\subsection{Missing Information}
	最大似然估计的渐进方差由费希尔(Fisher)信息量决定,根据第\ref{Score Statistic}节的定义,费希尔信息量为:
	\begin{equation}\label{fisher}
		\mathcal{F}=E\{S(\boldsymbol{\theta})S(\boldsymbol{\theta})^T \mid \mathbf{y},\boldsymbol{\theta} \}=E\{J(\boldsymbol{\theta})\mid\mathbf{y}, \boldsymbol{\theta}\}
	\end{equation}
	其中$J(\boldsymbol{\theta})=-\frac{\partial^2 log f(\mathbf{y} \mid \boldsymbol{\theta})}{\partial \boldsymbol{\theta}\partial \boldsymbol{\theta}^T}$,令
	$J_c(\boldsymbol{\theta})=-\frac{\partial^2 log f(\mathbf{y}, \mathbf{s} \mid \boldsymbol{\theta})}{\partial \boldsymbol{\theta}\partial \boldsymbol{\theta}^T}$,$\mathcal{F}_c=E\{J_c\boldsymbol{\theta})\mid \mathbf{y}, \boldsymbol{\theta}\}$. 由\eqref{incomplete data decomposition}得:	
	$$
	\log f(\mathbf{y} \mid \boldsymbol{\theta}) =\log f(\mathbf{y}, \mathbf{s} \mid \boldsymbol{\theta})-\log f(\mathbf{s} \mid \mathbf{y},\boldsymbol{\theta}),	
	$$
	对上面等式关于参数$\boldsymbol{\theta}$同时求二次导,有:
	$$
		J(\boldsymbol{\theta})=J_c(\boldsymbol{\theta})+\frac{\partial^2 log f( \mathbf{s} \mid \mathbf{y},\boldsymbol{\theta})}{\partial \boldsymbol{\theta}\partial \boldsymbol{\theta}^T},
	$$
	两边求条件期望得:
	\begin{equation}\label{conditional expectation}
		E\{J(\boldsymbol{\theta}) \mid \mathbf{y}, \boldsymbol{\theta}\}=E\{J_c(\boldsymbol{\theta}) \mid \mathbf{y}, \boldsymbol{\theta}\}+E\{\frac{\partial^2 log f( \mathbf{s} \mid \mathbf{y},\boldsymbol{\theta})}{\partial \boldsymbol{\theta}\partial \boldsymbol{\theta}^T} \mid \mathbf{y}, \boldsymbol{\theta}\},	
	\end{equation}
	
	观察等式左边,有
	$$
	\begin{aligned}
		E\{J(\boldsymbol{\theta}) \mid \mathbf{y}, \boldsymbol{\theta}\} & = \sum_s J(\boldsymbol{\theta})\cdot f(\mathbf{s} \mid \mathbf{y}, \boldsymbol{\theta}) 
		\\& = \sum_s-\frac{\partial^2 log f(\mathbf{y} \mid \boldsymbol{\theta})}{\partial \boldsymbol{\theta}\partial \boldsymbol{\theta}^T}\cdot f(\mathbf{s} \mid \mathbf{y}, \boldsymbol{\theta})
		\\ & = J(\boldsymbol{\theta})\cdot \sum_s f(\mathbf{s} \mid \mathbf{y}, \boldsymbol{\theta}) 
		\\& = J(\boldsymbol{\theta})\cdot 1 \\
		& = J(\boldsymbol{\theta})
	\end{aligned}  
	$$
	则我们可以简化\eqref{conditional expectation}表示为:
	\begin{equation}\label{error information}
	\underbrace{J(\boldsymbol{\theta})}_{\text{观测信息}}	=\underbrace{E\{J_c(\boldsymbol{\theta}) \mid \mathbf{y}, \boldsymbol{\theta}\}}_{\text{条件期望完整信息}}-\underbrace{J_m(\boldsymbol{\theta})}_{\text{缺失信息}}
	\end{equation}
	其中$J_m(\boldsymbol{\theta})=E\{\frac{\partial^2 log f( \mathbf{s} \mid \mathbf{y},\boldsymbol{\theta})}{\partial \boldsymbol{\theta}\partial \boldsymbol{\theta}^T} \mid \mathbf{y}, \boldsymbol{\theta}\}$,可以发现存在以下关系:
	$$
	\begin{WithArrows}
		-J_m(\boldsymbol{\theta})&=cov\{S_c(\boldsymbol{\theta})\mid \mathbf{y}, \boldsymbol{\theta}\}\\
		&=E\{S_c(\boldsymbol{\theta})S_c(\boldsymbol{\theta})^T \mid \mathbf{y}, \boldsymbol{\theta} \}-(E\{S_c(\boldsymbol{\theta}) \mid \mathbf{y}, \boldsymbol{\theta}\})^2\Arrow{根据\eqref{S and conditional Sc}}\\
		&=E\{S_c(\boldsymbol{\theta})S_c(\boldsymbol{\theta})^T \mid \mathbf{y}, \boldsymbol{\theta} \}-S(\boldsymbol{\theta})S(\boldsymbol{\theta})^T	
	\end{WithArrows}
	$$
	则\eqref{conditional expectation}可以表示为:
	\begin{DispWithArrows}[tikz=blue,tagged-lines=first]
		J(\boldsymbol{\theta})&=E\{J_c(\boldsymbol{\theta}) \mid \mathbf{y}, \boldsymbol{\theta}\}-E\{S_c(\boldsymbol{\theta})S_c(\boldsymbol{\theta})^T \mid \mathbf{y}, \boldsymbol{\theta} \}+S(\boldsymbol{\theta})S(\boldsymbol{\theta})^T\\
		&=E\{-\frac{\partial^2 log f(\mathbf{y}, \mathbf{s} \mid \boldsymbol{\theta})}{\partial \boldsymbol{\theta}\partial \boldsymbol{\theta}^T} \mid \mathbf{y}, \boldsymbol{\theta}\}-E\{S_c(\boldsymbol{\theta})S_c(\boldsymbol{\theta})^T \mid \mathbf{y}, \boldsymbol{\theta} \}+S(\boldsymbol{\theta})S(\boldsymbol{\theta})^T \Arrow{求和再求导等价于\\求导再求和}\\
		&=-\frac{\partial^2 Q\left(\boldsymbol{\theta}, \hat{\boldsymbol{\theta}}\right)}{\partial \boldsymbol{\theta}\partial \boldsymbol{\theta}^T}-E\{S_c(\boldsymbol{\theta})S_c(\boldsymbol{\theta})^T \mid \mathbf{y}, \boldsymbol{\theta} \}+\underbrace{S(\boldsymbol{\theta})S(\boldsymbol{\theta})^T}_{\text{最优时,一般约等于0}}
	\end{DispWithArrows}
	
	\subsection{Toy Example: 求解混合分布参数}
	
	\begin{ascolorbox1}[求解混合分布参数]{Toy Example}
		如下数据:
		\begin{align*}
			&3.54, 3.90, 3.93, 5.19, 3.58, 4.60, 3.85, 4.69, 4.29, 4.067, \\
			&3.77, 3.45, 5.36, 2.62, 4.80, 4.65, 3.65, 3.67, 6.23, 3.35, \\
			&1.58, 0.19, -1.89, 0.08, 0.34, 0.90, -0.03, 0.55, -0.57, -1.20
		\end{align*}
		可能来自于正态分布 $N(0,1)$ 与 $N(\mu,1)$ 的混合,混合比为 $1-p$ 与 $p$,且 $0<p<1$。求出 $p$ 与 $\mu$ 的极大似然估计.
	\end{ascolorbox1}
	
	首先给出混合密度:
	$$f(y;p,\mu)=p\phi(y-\mu)+(1-p)\phi(y)$$
	 其中$\phi(x)=\frac{1}{\sqrt{2\pi}}e^{-\frac{1}{2}x^2}$,设从混合分布中抽取样本$Y=(Y_1,Y_2,\cdots,Y_n)$,得到其似然函数:
	$$L(\mu,p,Y)=\prod_{i=1}^{n}(p\phi(Y_i-\mu)+(1-p)\phi(Y_i)).$$
	对数后
	$$l(\mu,p;Y)=\sum_{i=1}^{n}log(p\phi(Y_i-\mu)+(1-p)\phi(Y_i))$$
	 运用EM算法求解:引入潜在变量$Z=(z_1,z_2,\cdots,z_n)$,且$z_1,z_2,\cdots,z_n$相互独立,其中:
	$$Z_i=
	\left\{\begin{matrix}
		1& N(\mu,1)\\
		0& N(0,1)
	\end{matrix}\right.
	$$
	 以及$P(Z_i=1)=p,i=1,2,\cdots,n$,有$Y_i|Z_i=1\sim N(\mu,1)$,$Y_i|Z_i=0\sim N(0,1)$,则$(Z_i,Y_i)$的似然函数为:
	$$L(\mu,p;Y,Z)=\prod_{i=1}^{n}p^{Z_i}\phi(Y_i-\mu)^{Z_i}\phi(Y_i)^{1-Z_i}$$
	对上述似然函数取对数并去掉与$p$、$\mu$无关的量得:
	$$l_1(\mu,p;Y,Z)=\sum_{i=1}^{n}Z_ilogp-\frac{1}{2}\sum_{i=1}^{n}Z_i(Y_i-\mu)^2+(n-\sum_{i=1}^{n}Z_i)log(1-p).$$
	假设在第$k$步迭代中,有估计值$\mu^{k}$、$p^{k}$,通过E步和M步得到$\mu$、$p$的新的估计值$\mu^{(k+1)}$、$p^{(k+1)}$.
	
	在E步中,令:
	\begin{align*}
		Q(\mu,p|\mu^{(k)},p^{(k)},Y)
		&=E_Z[l_1(\mu,p;Y,Z)|\mu^{(k)},p^{(k)},Y]\\
		&=\sum_{i=1}^{n}E_Z[Z_i|\mu^{(k)},p^{(k)},Y]logp\\
		&-\frac{1}{2}\sum_{i=1}^{n}E_Z[Z_i|\mu^{(k)},p^{(k)},Y](Y_i-\mu)^2\\
		&+(n-\sum_{i=1}^{n}E_Z[Z_i|\mu^{(k)},p^{(k)},Y]log(1-p))
	\end{align*}
	易知:
	$$Z_i^{(K+1)}=E_Z[Z_i|\mu^{(k)},p^{(k)},Y]=\frac{p^{(k)}\phi(Y_i-\mu^{(k)})}{p^{(k)}\phi(Y_i-\mu^{(k)})+(1-p^{(k)})\phi(Y_i)}$$
	
	\begin{mybox1}
		根据贝叶斯定理,我们有:
		\[ P(Z_i = 1 | Y_i, \mu^{(k)}, p^{(k)}) = \frac{P(Y_i | Z_i = 1, \mu^{(k)}) P(Z_i = 1 | \mu^{(k)}, p^{(k)})}{P(Y_i | \mu^{(k)}, p^{(k)})} \]
		
		这里:
		\begin{itemize}
			\item $P(Y_i | Z_i = 1, \mu^{(k)})$ 是在$Z_i = 1$条件下的$Y_i$的概率密度,即$\phi(Y_i - \mu^{(k)})$.
			\item $P(Z_i = 1 | \mu^{(k)}, p^{(k)})$ 是$Z_i = 1$的先验概率,即$p^{(k)}$.
			\item $P(Y_i | \mu^{(k)}, p^{(k)})$ 是$Y_i$的总概率,它等于两种情况的加权和,即$p^{(k)}\phi(Y_i - \mu^{(k)}) + (1 - p^{(k)})\phi(Y_i)$.
		\end{itemize}
		
		因此,	
		\[ Z_i^{(k+1)} = E[Z_i|\mu^{(k)},p^{(k)},Y_i] = \frac{p^{(k)}\phi(Y_i - \mu^{(k)})}{p^{(k)}\phi(Y_i - \mu^{(k)}) + (1 - p^{(k)})\phi(Y_i)} \]		
	\end{mybox1}
	
	在M步中,解:
	$$
	\left\{\begin{matrix}
		\frac{\partial Q}{\partial \mu}=0\\
		\frac{\partial Q}{\partial p}=0
	\end{matrix}\right.
	$$
	求得:
	$$
	\mu^{(k+1)}=\frac{\sum_{i=1}^{n}\frac{\phi(Y_i-\mu^{(k)})Y_i}{p^{(k)}\phi(Y_i-\mu^{(k)})+(1-p^{(k)})\phi(Y_i)}}{\sum_{i=1}^{n}\frac{\phi(Y_i-\mu^{(k)})}{p^{(k)}\phi(Y_i-\mu^{(k)})+(1-p^{(k)})\phi(Y_i)}};
	$$
	
	$$
		p^{(k+1)}=\frac{1}{n}\sum_{i=1}^{n}\frac{p^{(k)}\phi(Y_i-\mu^{(k)})}{p^{(k)}\phi(Y_i-\mu^{(k)})+(1-p^{(k)})\phi(Y_i)}.
	$$
	
	\section{Example: MULTINOMIAL WITH COMPLEX CELL STRUCTURE\label{cell example}}

	假设我们有$n=435$次观测,观测对象是一个有四个遗传性状的多项式分布,这些性状的概率结构如表\ref{cell}所示. 表中还给出了这些性状观测到的频率. (其中基因为$O$、$A$以及$B$,三个基因相互组合,通过观测值求解每个基因对应的概率$r$、$A$和$B$.)
	\subsection{MLE Method}
	\begin{table}[H]
		\caption{观测细胞数据}
		\label{cell}
		\centering
		\begin{tabular}{ccc}
			
			\hline \begin{tabular}{c} 
				类别 \\
				(细胞)
			\end{tabular} & \begin{tabular}{c} 
				细胞 \\
				概率
			\end{tabular} & \begin{tabular}{c} 
				观测 \\
				频率
			\end{tabular} \\
			\hline $\mathrm{O}$ & $r^2$ & $n_O=176$ \\
			$\mathrm{~A}$ & $p^2+2 p r$ & $n_A=182$ \\
			$\mathrm{~B}$ & $q^2+2 q r$ & $n_B=60$ \\
			$\mathrm{AB}$ & $2 p q$ & $n_{A B}=17$ \\
			\hline
			
		\end{tabular}
	\end{table}
	因此,观测数据由性状频率的向量给出:
	$$
	\boldsymbol{y}=\left(n_O, n_A, n_B, n_{A B}\right)^T .
	$$
	未知参数的向量为:
	$$
	\Psi=(p, q)^T,
	$$
	因为 $r=1-p-q$. 目标是基于$\boldsymbol{y}$找到$\Psi$的最大似然估计(MLE). 这是遗传学中基因频率估计的一个著名问题,很多研究都有讨论.
	
	参数$\boldsymbol{\Psi}$的对数似然函数(除了一个加性常数之外)为
	$$
	\log L(\Psi)=2 n_0 \log  \underbrace{r}_{\pi_1}+n_A \log \underbrace{\left(p^2+2 p r\right)}_{\pi_2} +n_B \log \underbrace{\left(q^2+2 q r\right)}_{\pi_3} +n_{A B} \log \underbrace{(2 p q)}_{\pi_4},
	$$
	它没有一个封闭形式的解决方案来获得$\hat{\boldsymbol{\Psi}}$,即$\Psi$的最大似然估计(MLE).
	
	我们将单元频率表示为$\pi_j(j=1,2,3,4)$。那么它们关于$\Psi$的一阶和二阶导数如下:
	$$
	\begin{array}{cc}
		\frac{\partial \pi_1(\boldsymbol{\Psi})}{\partial \boldsymbol{\Psi}}=\left(\begin{array}{c}
			-2 r \\
			-2 r
		\end{array}\right) ; & \frac{\partial \pi_2(\boldsymbol{\Psi})}{\partial \boldsymbol{\Psi}}=\left(\begin{array}{r}
		   2p+2r \\
			-2 p
		\end{array}\right) \\
		\frac{\partial \pi_3(\boldsymbol{\Psi})}{\partial \boldsymbol{\Psi}}=\left(\begin{array}{r}
			2 q+2r \\
			-2 q
		\end{array}\right) ; & \frac{\partial \pi_4(\boldsymbol{\Psi})}{\partial \boldsymbol{\Psi}}=\left(\begin{array}{c}
			2 q \\
			2 p
		\end{array}\right) \\
		\frac{\partial^2 \pi_1(\boldsymbol{\Psi})}{\partial \boldsymbol{\Psi} \partial \boldsymbol{\Psi}^T}=\left(\begin{array}{cc}
			-2 & -2 \\
			-2 & -2
		\end{array}\right) ; & \frac{\partial^2 \pi_2(\boldsymbol{\Psi})}{\partial \boldsymbol{\Psi} \partial \boldsymbol{\Psi}^T}=\left(\begin{array}{rr}
			0 & -2 \\
			-2 & 0
		\end{array}\right) \\
		\frac{\partial^2 \pi_3(\boldsymbol{\Psi})}{\partial \boldsymbol{\Psi} \partial \boldsymbol{\Psi}^T}=\left(\begin{array}{rr}
			-2 & 0 \\
			2 & -2
		\end{array}\right) ; & \frac{\partial^2 \pi_4(\boldsymbol{\Psi})}{\partial \boldsymbol{\Psi} \partial \boldsymbol{\Psi}^T}=\left(\begin{array}{cc}
			0 & 2 \\
			2 & 0
		\end{array}\right) .
	\end{array}
	$$
	
	这导致了如下的似然方程:
	$$
	\partial \log L(\boldsymbol{\Psi}) / \partial \boldsymbol{\Psi}=\sum_{j=1}^4\left(\frac{n_j}{\pi_j}\right) \frac{\partial \pi_j(\boldsymbol{\Psi})}{\partial \boldsymbol{\Psi}}=\mathbf{0},
	$$
	以及对数似然的Hessian矩阵:
	$$
	\partial^2 \log L(\boldsymbol{\Psi}) / \partial \boldsymbol{\Psi} \partial \boldsymbol{\Psi}^T=\sum_{j=1}^4 n_j\left\{\left(\frac{1}{\pi_j}\right) \frac{\partial^2 \pi_j(\boldsymbol{\Psi})}{\partial \boldsymbol{\Psi} \boldsymbol{\Psi}^T}-\left(\frac{1}{\pi_j^2}\right) \frac{\partial \pi_j(\boldsymbol{\Psi})}{\partial \boldsymbol{\Psi}}\left(\frac{\partial \pi_j(\boldsymbol{\Psi})}{\partial \boldsymbol{\Psi}^T}\right)\right\} .
	$$
	
	费舍尔(预期)信息矩阵由以下公式给出:
	$$
	\begin{aligned}
		\mathcal{I}(\boldsymbol{\Psi}) & =E\left\{-\partial^2 \log L(\boldsymbol{\Psi}) / \partial \boldsymbol{\Psi} \partial \boldsymbol{\Psi}^T\right\} \\
		& =n\left\{\sum_{j=1}^4\left(\frac{1}{\pi_j}\right) \frac{\partial \pi_j(\boldsymbol{\Psi})}{\partial \boldsymbol{\Psi}}\left(\frac{\partial \pi_j(\boldsymbol{\Psi})}{\partial \boldsymbol{\Psi}^T}\right)\right\},
	\end{aligned}
	$$
	当条件设置为 \(\boldsymbol{\Psi}=\hat{\boldsymbol{\Psi}}\) 时,所得到的协方差矩阵为:
	\[
	\left(\begin{array}{rr}
		0.000011008 & -0.000103688 \\
		-0.000103688 & 0.000040212
	\end{array}\right) ;
	\]
	参见\citet{monahan2011numerical}对这个例子中牛顿方法、评分方法和EM算法的有趣讨论.
	
	\subsection{EM Method}
	现在让我们讨论将EM算法应用于这个问题. 在将EM算法应用于这个问题时,一个自然的选择是完整数据向量为:
	\[
	\boldsymbol{x}=\left(n_O, \boldsymbol{z}^T\right)^T,
	\]
	其中
	\[
	\boldsymbol{z}=\left(n_{A A}, n_{A O}, n_{B B}, n_{B O}\right)^T
	\]
	表示不可观测或“缺失”的数据。这些数据被认为是频率 \(n_{AA}, n_{AO}, n_{BB}\)(因为这些基因显现出来是A,实则可能是AO或者AA),和 \(n_{BO}\),对应于表格 2.7 中的中间单元格。值得注意的是,由于总频率 \(n\) 是固定的,因此变量$\boldsymbol{x}$中的五个单元格频率足以代表完整数据. 如果我们认为 \(\boldsymbol{x}\) 的分布是关于表格 \ref{Complete-Data Structure}中指定的六个单元格概率的 \(n\) 次抽取的多项式分布,那么很明显,观察到的频率向量 \(y\) 具有所需的多项式分布,如表\ref{cell}所指定.
		\begin{table}[H]
		\caption{Complete-Data Structure for Example } 
		\label{Complete-Data Structure}
		\centering
		\begin{tabular}{ccc}
			\hline \begin{tabular}{c} 
				Category \\
				$($ Cell $)$
			\end{tabular} & \begin{tabular}{c} 
				Cell \\
				Probability
			\end{tabular} & \begin{tabular}{c} 
				Notation for \\
				Frequency
			\end{tabular} \\
			\hline $\mathrm{O}$ & $r^2$ & $n_O$ \\
			$\mathrm{AA}$ & $p^2$ & $n_{A A}$ \\
			$\mathrm{AO}$ & $2 p r$ & $n_{A O}$ \\
			$\mathrm{BB}$ & $q^2$ & $n_{B B}$ \\
			$\mathrm{BO}$ & $2 q r$ & $n_{B O}$ \\
			$\mathrm{AB}$ & $2 p q$ & $n_{A B}$ \\
			\hline
		\end{tabular}
	\end{table}

	对于 \(\boldsymbol{\Psi}\),完整数据的对数似然函数可以写成(除了一个加法常数)如下形式:
	\begin{equation}\label{cell likelihood}
		\log L_c(\boldsymbol{\Psi})=2 n_A^{+} \log p+2 n_B^{+} \log q+2 n_O^{+} \log r,
	\end{equation}
	其中
	\[
	\begin{aligned}
		& n_A^{+}=n_{AA}+\frac{1}{2} n_{AO}+\frac{1}{2} n_{AB}, \\
		& n_B^{+}=n_{BB}+\frac{1}{2} n_{BO}+\frac{1}{2} n_{AB},
	\end{aligned}
	\]
	和
	\[
	n_O^{+}=n_O+\frac{1}{2} n_{AO}+\frac{1}{2} n_{BO} .
	\]
	在这里,\(\log L_c(\boldsymbol{\Psi})\) 表示的是完整数据的对数似然函数,它是用于估计统计模型参数的一个关键函数.
	
	公式\eqref{cell likelihood}呈现的是关于频率 \(2n_A^{+}, 2n_B^{+}\) 和 \(2n_O^{+}\) 的多项式对数似然函数,对应于三个单元格的概率 \(p, q\) 和 \(r\). 因此,通过最大化\eqref{cell likelihood} 得到这些概率的完整数据最大似然估计(MLE)如下:
	\begin{equation}\label{cell EM MLE}
		\hat{p}=\frac{n_A^{+}}{n} ; \quad \hat{q}=\frac{n_B^{+}}{n} .
	\end{equation}
	当完整数据似然函数属于规则指数族时,E步骤和M步骤会简化,就像这个例子一样. E步骤仅需要计算当前条件下 \(\boldsymbol{\Psi}\) 的充分统计量的期望值,这里是 \(\left(n_A^{+}, n_B^{+}\right)^T\). M步骤随后通过解由等同于这个期望的方程得到 \(\boldsymbol{\Psi}^{(k+1)}\). 对于这个问题,实际上 \(\boldsymbol{\Psi}^{(k+1)}\) 是通过用观察到的数据给定的 \(n_A^{+}\) 和 \(n_B^{+}\) 的当前条件期望来替换\eqref{cell EM MLE}右侧的值来得到的.
	
	为了计算 \(n_A^{+}\) 和 \(n_B^{+}\)(即E步)的这些条件期望,我们需要计算这个问题中不可观察数据 \(z\) 的条件期望. 考虑 \(z\) 的第一个元素,从变量$\boldsymbol{x}$可知是 \(n_{AA}\).
	
	首先,可以验证,在给定 \(\boldsymbol{y}\) 的条件下,\(n_A, n_{AA}\) 实际上具有二项分布,样本大小为 \(n_A\),概率参数为
	\[
	p^{(k)^2} /\left(p^{(k)^2}+2 p^{(k)} r^{(k)}\right),
	\]
	\begin{ascolorbox16}{Why 服从二项分布}
		在迭代的第$k$步,根据贝叶斯定理,有
		$$
		\begin{aligned}
			&p\left(\boldsymbol{z}^{(k+1)}=A A \mid \mathbf{y} =A, \boldsymbol{\Psi}^{(k)}\right)
			=\frac{p\left(\mathbf{y}=A \mid \boldsymbol{z}^{(k)}=A A, \boldsymbol{\Psi}^{(k)}\right) \cdot P\left(\boldsymbol{z}^{(k)}=A A \mid \boldsymbol{\Psi}^{(k)}\right)}{p\left(\mathbf{y}=A \mid \boldsymbol{z}^{(k)}=A A, \boldsymbol{\Psi}^{(k)}\right)}\\
			&= \frac{p\left(\mathbf{y}=A \mid \boldsymbol{z}^{(k)}=A A, \boldsymbol{\Psi}^{(k)}\right) \cdot P\left(\boldsymbol{z}^{(k)}=A A \mid \boldsymbol{\Psi}^{(k)}\right)}{p\left(\mathbf{y}=A,\boldsymbol{z}^{(k)}=A A\mid  \boldsymbol{\Psi}^{(k)}\right)+p\left(\mathbf{y}=A,\boldsymbol{z}^{(k)}=AO \mid  \boldsymbol{\Psi}^{(k)}\right)}\\
			&=\frac{p^{(k)^2}\cdot p\left(\mathbf{y}=A \mid \boldsymbol{z}^{(k)}=A A, \boldsymbol{\Psi}^{(k)}\right)}{(p^{(k)^2}+2 p^{(k)} r^{(k)})\cdot p\left(\mathbf{y}=A \mid \boldsymbol{z}^{(k)}=A A, \boldsymbol{\Psi}^{(k)}\right)}\\
			&=\frac{p^{(k)^2}}{p^{(k)^2}+2 p^{(k)} r^{(k)}}.
		\end{aligned}
		$$		
	\end{ascolorbox16}
	这里 \(\boldsymbol{\Psi}^{(k)}\) 代替了未知参数向量 \(\boldsymbol{\Psi}\) 在第 \(k+1\) 次迭代中的使用. 因此,给定 \(y\) 的 \(n_{AA}\) 的当前条件期望可以通过
	\[
	E_{\boldsymbol{\Psi}^{(k)}}\left(n_{AA}\right)=n_{AA}^{(k)},
	\]
	得到,其中
	\begin{equation}\label{naa}
		n_{AA}^{(k)}=n_A p^{(k)^2} /\left(p^{(k)^2}+2 p^{(k)} r^{(k)}\right) .
	\end{equation}
	
	同样,给定 \(\boldsymbol{y}\) 的 \(n_{AO}, n_{BB}\)和 \(n_{BO}\) 的当前条件期望也可以计算出来. M步骤的执行给出
	\[
	p^{(k+1)}=\left(n_{AA}^{(k)}+\frac{1}{2} n_{AO}^{(k)}+\frac{1}{2} n_{AB}\right) / n
	\]
	和
	\[
	q^{(k+1)}=\left(n_{BB}^{(k)}+\frac{1}{2} n_{BO}^{(k)}+\frac{1}{2} n_{AB}\right) / n .
	\]
	这个问题的EM算法结果在表\ref{EM results for cell}中给出。可以将 \(\boldsymbol{\Psi}\) 的最大似然估计(MLE)视为第 \(k=4\) 次迭代时 \(\Psi^{(k)}\) 的值.
	
	\begin{table}[H]
		\caption{Results of the EM Algorithm for Example.}
		\label{EM results for cell}
		\centering 
		\begin{tabular}{ccccr}
			\hline Iteration & $p^{(k)}$ & $q^{(k)}$ & $r^{(k)}$ & $-\log L\left(\boldsymbol{\Psi}^{(k)}\right)$ \\
			\hline 0 & 0.26399 & 0.09299 & 0.64302 & 2.5619001 \\
			1 & 0.26436 & 0.09316 & 0.64248 & 2.5577875 \\
			2 & 0.26443 & 0.09317 & 0.64240 & 2.5577729 \\
			3 & 0.26444 & 0.09317 & 0.64239 & 2.5577726 \\
			4 & 0.26444 & 0.09317 & 0.64239 & 2.5577726 \\
			\hline
		\end{tabular}
	\end{table}
	
	
	\section{Monte Carlo Versions of the EM Algorithm}
	\subsection{MONTE CARLO EM}
	在EM算法中,E步骤可能难以实施,因为难以计算对数似然的期望值.\citet{wei1990monte,wei1990posterior}建议采用蒙特卡洛方法,通过在第$(k+1)$次迭代的E步骤中从条件分布$ f(\mathbf{s} \mid \mathbf{y}, \boldsymbol{\theta}) $模拟缺失数据$\mathbf{s}$,然后最大化完全数据对数似然的近似条件期望:
	\begin{equation}\label{MCEM Q function}
		\hat{Q}\left(\boldsymbol{\theta};\hat{\boldsymbol{\theta}} \right)=\frac{1}{m} \sum_{i=1}^{m} \log \left[f\left(\mathbf{y}, \mathbf{s}^{(i)} \mid \boldsymbol{\theta}\right)\right],
	\end{equation}
	当$m \rightarrow \infty$时,这个公式的极限形式就是实际的$Q\left(\boldsymbol{\theta};\hat{\boldsymbol{\theta}} \right)$. 这正是蒙特卡洛积分的核心思想. 尽管最大化公式\eqref{MCEM Q function}通常可能很困难,但有时,在指数族情形下,最大化问题可以有解析形式的解. 
	\begin{marker}
		在MCEM(Monte Carlo Expectation-Maximization)中,蒙特卡洛误差在E步骤引入,丧失了单调性属性. 但在某些情况下,该算法以很高的概率接近一个极大化值.
	\end{marker}
	\subsection{Estimation of Standard Error with MCEM}
	最大似然估计(MLE)$\hat{\boldsymbol{\theta}}$的协方差矩阵估计由观测信息矩阵$J(\hat{\boldsymbol{\theta}})$的逆给出,根据公式\eqref{error information},
	\begin{equation}\label{MCEM ERROR}
		J(\boldsymbol{\theta})	=E\{J_c(\boldsymbol{\theta}) \mid \mathbf{y}, \boldsymbol{\theta}\}-J_m(\boldsymbol{\theta}),
	\end{equation}
	其中
	$$
	J_c(\boldsymbol{\theta})=-\frac{\partial^2 log f(\mathbf{y}, \mathbf{s} \mid \boldsymbol{\theta})}{\partial \boldsymbol{\theta}\partial \boldsymbol{\theta}^T},
	$$
	
	现在考虑单一未知参数$\boldsymbol{\theta}$的情况,并将完全数据对数似然函数$\log f(\mathbf{y}, \mathbf{s} \mid \boldsymbol{\theta})$写为$\log f(\mathbf{y}, \mathbf{s}^{(i)} \mid \boldsymbol{\theta})$. 为了通过蒙特卡洛评估计算\eqref{MCEM ERROR}中的期望值,我们可以将$J(\boldsymbol{\theta})$表示为以下形式:
	$$
	\begin{aligned}
		& J(\boldsymbol{\theta}) \approx \frac{1}{m} \sum_{j=\mathbf{1}}^m-\partial^2 \log f(\mathbf{y}, \mathbf{s}^{(i)} \mid \boldsymbol{\theta}) / \partial \boldsymbol{\theta}^2 \\
		&+\frac{1}{m} \sum_{j=\mathbf{1}}^m\left\{\partial \log f(\mathbf{y}, \mathbf{s}^{(i)} \mid \boldsymbol{\theta}) / \partial \boldsymbol{\theta}-\frac{1}{m} \sum_{j=\mathbf{1}}^m \partial \log f(\mathbf{y}, \mathbf{s}^{(i)} \mid \boldsymbol{\theta}) / \partial \boldsymbol{\theta}\right\}^2.
	\end{aligned}
	$$
	其中$\mathbf{s}^{(j)}(j=1, \ldots, m)$是从缺失数据分布生成的,使用MCEM估计的$\boldsymbol{\theta}$.
	\subsection{Continue to Section \ref{cell example}}
	在例子Section \ref{cell example} 中,如果我们采用蒙特卡罗(MC)期望步骤(E-step),我们可以分别从两个独立的二项分布中抽取 \(z_{11}, \ldots, z_{1m}\) 和 \(z_{21}, \ldots, z_{2m}\),其中第一个二项分布的样本大小为 \(n_A\),概率参数为
	\[
	p^{(k)^2} /\left(p^{(k)^2}+2 p^{(k)} r^{(k)}\right),
	\]
	而第二个二项分布的样本大小为 \(n_B\),概率参数为
	\[
	q^{(k)^2} /\left(q^{(k)^2}+2 q^{(k)} r^{(k)}\right),
	\]
	在第 \(k+1\) 次迭代中,用 \(\boldsymbol{\Psi}^{(k)}\) 替代未知的参数向量 \(\boldsymbol{\Psi}\). 然后,这些抽取的值可以代替方程\eqref{naa}使用,如下所示:
	\[
	n_{AA}^{(k)}=\bar{z}_{1m}=\frac{1}{m} \sum_{j=1}^m z_{1i}, \quad n_{BB}^{(k)}=\bar{z}_{2m}=\frac{1}{m} \sum_{j=1}^m z_{2i} .
	\]
	在这个例子中,通过使用蒙特卡罗方法模拟缺失数据 \(z\) 的可能值,我们可以获得对 \(n_{AA}\) 和 \(n_{BB}\) 的估计。这是通过在每个迭代中从相应的二项分布中抽取样本来实现的,然后计算这些抽取值的平均数来估计 \(n_{AA}\) 和 \(n_{BB}\). 这种方法允许我们在存在缺失或不完整数据时,使用模拟数据来估计缺失值,从而在EM算法中实施期望步骤.
	%参考文献
	%-------------------------------------------
	\newpage
	\bibliographystyle{plainnat}%
	\bibliography{refs.bib}
\end{document}